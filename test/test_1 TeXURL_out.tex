\documentclass[11pt]{article}

    \usepackage[breakable]{tcolorbox}
    \usepackage{parskip} % Stop auto-indenting (to mimic markdown behaviour)
    
    \usepackage{iftex}
    \ifPDFTeX
    	\usepackage[T1]{fontenc}
    	\usepackage{mathpazo}
    \else
    	\usepackage{fontspec}
    \fi

    % Basic figure setup, for now with no caption control since it's done
    % automatically by Pandoc (which extracts ![](path) syntax from Markdown).
    \usepackage{graphicx}
    % Maintain compatibility with old templates. Remove in nbconvert 6.0
    \let\Oldincludegraphics\includegraphics
    % Ensure that by default, figures have no caption (until we provide a
    % proper Figure object with a Caption API and a way to capture that
    % in the conversion process - todo).
    \usepackage{caption}
    \DeclareCaptionFormat{nocaption}{}
    \captionsetup{format=nocaption,aboveskip=0pt,belowskip=0pt}

    \usepackage{float}
    \floatplacement{figure}{H} % forces figures to be placed at the correct location
    \usepackage{xcolor} % Allow colors to be defined
    \usepackage{enumerate} % Needed for markdown enumerations to work
    \usepackage{geometry} % Used to adjust the document margins
    \usepackage{amsmath} % Equations
    \usepackage{amssymb} % Equations
    \usepackage{textcomp} % defines textquotesingle
    % Hack from http://tex.stackexchange.com/a/47451/13684:
    \AtBeginDocument{%
        \def\PYZsq{\textquotesingle}% Upright quotes in Pygmentized code
    }
    \usepackage{upquote} % Upright quotes for verbatim code
    \usepackage{eurosym} % defines \euro
    \usepackage[mathletters]{ucs} % Extended unicode (utf-8) support
    \usepackage{fancyvrb} % verbatim replacement that allows latex
    \usepackage{grffile} % extends the file name processing of package graphics 
                         % to support a larger range
    \makeatletter % fix for old versions of grffile with XeLaTeX
    \@ifpackagelater{grffile}{2019/11/01}
    {
      % Do nothing on new versions
    }
    {
      \def\Gread@@xetex#1{%
        \IfFileExists{"\Gin@base".bb}%
        {\Gread@eps{\Gin@base.bb}}%
        {\Gread@@xetex@aux#1}%
      }
    }
    \makeatother
    \usepackage[Export]{adjustbox} % Used to constrain images to a maximum size
    \adjustboxset{max size={0.9\linewidth}{0.9\paperheight}}

    % The hyperref package gives us a pdf with properly built
    % internal navigation ('pdf bookmarks' for the table of contents,
    % internal cross-reference links, web links for URLs, etc.)
    \usepackage{hyperref}
    % The default LaTeX title has an obnoxious amount of whitespace. By default,
    % titling removes some of it. It also provides customization options.
    \usepackage{titling}
    \usepackage{longtable} % longtable support required by pandoc >1.10
    \usepackage{booktabs}  % table support for pandoc > 1.12.2
    \usepackage[inline]{enumitem} % IRkernel/repr support (it uses the enumerate* environment)
    \usepackage[normalem]{ulem} % ulem is needed to support strikethroughs (\sout)
                                % normalem makes italics be italics, not underlines
    \usepackage{mathrsfs}
    

    
    % Colors for the hyperref package
    \definecolor{urlcolor}{rgb}{0,.145,.698}
    \definecolor{linkcolor}{rgb}{.71,0.21,0.01}
    \definecolor{citecolor}{rgb}{.12,.54,.11}

    % ANSI colors
    \definecolor{ansi-black}{HTML}{3E424D}
    \definecolor{ansi-black-intense}{HTML}{282C36}
    \definecolor{ansi-red}{HTML}{E75C58}
    \definecolor{ansi-red-intense}{HTML}{B22B31}
    \definecolor{ansi-green}{HTML}{00A250}
    \definecolor{ansi-green-intense}{HTML}{007427}
    \definecolor{ansi-yellow}{HTML}{DDB62B}
    \definecolor{ansi-yellow-intense}{HTML}{B27D12}
    \definecolor{ansi-blue}{HTML}{208FFB}
    \definecolor{ansi-blue-intense}{HTML}{0065CA}
    \definecolor{ansi-magenta}{HTML}{D160C4}
    \definecolor{ansi-magenta-intense}{HTML}{A03196}
    \definecolor{ansi-cyan}{HTML}{60C6C8}
    \definecolor{ansi-cyan-intense}{HTML}{258F8F}
    \definecolor{ansi-white}{HTML}{C5C1B4}
    \definecolor{ansi-white-intense}{HTML}{A1A6B2}
    \definecolor{ansi-default-inverse-fg}{HTML}{FFFFFF}
    \definecolor{ansi-default-inverse-bg}{HTML}{000000}

    % common color for the border for error outputs.
    \definecolor{outerrorbackground}{HTML}{FFDFDF}

    % commands and environments needed by pandoc snippets
    % extracted from the output of `pandoc -s`
    \providecommand{\tightlist}{%
      \setlength{\itemsep}{0pt}\setlength{\parskip}{0pt}}
    \DefineVerbatimEnvironment{Highlighting}{Verbatim}{commandchars=\\\{\}}
    % Add ',fontsize=\small' for more characters per line
    \newenvironment{Shaded}{}{}
    \newcommand{\KeywordTok}[1]{\textcolor[rgb]{0.00,0.44,0.13}{\textbf{{#1}}}}
    \newcommand{\DataTypeTok}[1]{\textcolor[rgb]{0.56,0.13,0.00}{{#1}}}
    \newcommand{\DecValTok}[1]{\textcolor[rgb]{0.25,0.63,0.44}{{#1}}}
    \newcommand{\BaseNTok}[1]{\textcolor[rgb]{0.25,0.63,0.44}{{#1}}}
    \newcommand{\FloatTok}[1]{\textcolor[rgb]{0.25,0.63,0.44}{{#1}}}
    \newcommand{\CharTok}[1]{\textcolor[rgb]{0.25,0.44,0.63}{{#1}}}
    \newcommand{\StringTok}[1]{\textcolor[rgb]{0.25,0.44,0.63}{{#1}}}
    \newcommand{\CommentTok}[1]{\textcolor[rgb]{0.38,0.63,0.69}{\textit{{#1}}}}
    \newcommand{\OtherTok}[1]{\textcolor[rgb]{0.00,0.44,0.13}{{#1}}}
    \newcommand{\AlertTok}[1]{\textcolor[rgb]{1.00,0.00,0.00}{\textbf{{#1}}}}
    \newcommand{\FunctionTok}[1]{\textcolor[rgb]{0.02,0.16,0.49}{{#1}}}
    \newcommand{\RegionMarkerTok}[1]{{#1}}
    \newcommand{\ErrorTok}[1]{\textcolor[rgb]{1.00,0.00,0.00}{\textbf{{#1}}}}
    \newcommand{\NormalTok}[1]{{#1}}
    
    % Additional commands for more recent versions of Pandoc
    \newcommand{\ConstantTok}[1]{\textcolor[rgb]{0.53,0.00,0.00}{{#1}}}
    \newcommand{\SpecialCharTok}[1]{\textcolor[rgb]{0.25,0.44,0.63}{{#1}}}
    \newcommand{\VerbatimStringTok}[1]{\textcolor[rgb]{0.25,0.44,0.63}{{#1}}}
    \newcommand{\SpecialStringTok}[1]{\textcolor[rgb]{0.73,0.40,0.53}{{#1}}}
    \newcommand{\ImportTok}[1]{{#1}}
    \newcommand{\DocumentationTok}[1]{\textcolor[rgb]{0.73,0.13,0.13}{\textit{{#1}}}}
    \newcommand{\AnnotationTok}[1]{\textcolor[rgb]{0.38,0.63,0.69}{\textbf{\textit{{#1}}}}}
    \newcommand{\CommentVarTok}[1]{\textcolor[rgb]{0.38,0.63,0.69}{\textbf{\textit{{#1}}}}}
    \newcommand{\VariableTok}[1]{\textcolor[rgb]{0.10,0.09,0.49}{{#1}}}
    \newcommand{\ControlFlowTok}[1]{\textcolor[rgb]{0.00,0.44,0.13}{\textbf{{#1}}}}
    \newcommand{\OperatorTok}[1]{\textcolor[rgb]{0.40,0.40,0.40}{{#1}}}
    \newcommand{\BuiltInTok}[1]{{#1}}
    \newcommand{\ExtensionTok}[1]{{#1}}
    \newcommand{\PreprocessorTok}[1]{\textcolor[rgb]{0.74,0.48,0.00}{{#1}}}
    \newcommand{\AttributeTok}[1]{\textcolor[rgb]{0.49,0.56,0.16}{{#1}}}
    \newcommand{\InformationTok}[1]{\textcolor[rgb]{0.38,0.63,0.69}{\textbf{\textit{{#1}}}}}
    \newcommand{\WarningTok}[1]{\textcolor[rgb]{0.38,0.63,0.69}{\textbf{\textit{{#1}}}}}
    
    
    % Define a nice break command that doesn't care if a line doesn't already
    % exist.
    \def\br{\hspace*{\fill} \\* }
    % Math Jax compatibility definitions
    \def\gt{>}
    \def\lt{<}
    \let\Oldtex\TeX
    \let\Oldlatex\LaTeX
    \renewcommand{\TeX}{\textrm{\Oldtex}}
    \renewcommand{\LaTeX}{\textrm{\Oldlatex}}
    % Document parameters
    % Document title
    \title{Untitled}
    
    
    
    
    
% Pygments definitions
\makeatletter
\def\PY@reset{\let\PY@it=\relax \let\PY@bf=\relax%
    \let\PY@ul=\relax \let\PY@tc=\relax%
    \let\PY@bc=\relax \let\PY@ff=\relax}
\def\PY@tok#1{\csname PY@tok@#1\endcsname}
\def\PY@toks#1+{\ifx\relax#1\empty\else%
    \PY@tok{#1}\expandafter\PY@toks\fi}
\def\PY@do#1{\PY@bc{\PY@tc{\PY@ul{%
    \PY@it{\PY@bf{\PY@ff{#1}}}}}}}
\def\PY#1#2{\PY@reset\PY@toks#1+\relax+\PY@do{#2}}

\@namedef{PY@tok@w}{\def\PY@tc##1{\textcolor[rgb]{0.73,0.73,0.73}{##1}}}
\@namedef{PY@tok@c}{\let\PY@it=\textit\def\PY@tc##1{\textcolor[rgb]{0.25,0.50,0.50}{##1}}}
\@namedef{PY@tok@cp}{\def\PY@tc##1{\textcolor[rgb]{0.74,0.48,0.00}{##1}}}
\@namedef{PY@tok@k}{\let\PY@bf=\textbf\def\PY@tc##1{\textcolor[rgb]{0.00,0.50,0.00}{##1}}}
\@namedef{PY@tok@kp}{\def\PY@tc##1{\textcolor[rgb]{0.00,0.50,0.00}{##1}}}
\@namedef{PY@tok@kt}{\def\PY@tc##1{\textcolor[rgb]{0.69,0.00,0.25}{##1}}}
\@namedef{PY@tok@o}{\def\PY@tc##1{\textcolor[rgb]{0.40,0.40,0.40}{##1}}}
\@namedef{PY@tok@ow}{\let\PY@bf=\textbf\def\PY@tc##1{\textcolor[rgb]{0.67,0.13,1.00}{##1}}}
\@namedef{PY@tok@nb}{\def\PY@tc##1{\textcolor[rgb]{0.00,0.50,0.00}{##1}}}
\@namedef{PY@tok@nf}{\def\PY@tc##1{\textcolor[rgb]{0.00,0.00,1.00}{##1}}}
\@namedef{PY@tok@nc}{\let\PY@bf=\textbf\def\PY@tc##1{\textcolor[rgb]{0.00,0.00,1.00}{##1}}}
\@namedef{PY@tok@nn}{\let\PY@bf=\textbf\def\PY@tc##1{\textcolor[rgb]{0.00,0.00,1.00}{##1}}}
\@namedef{PY@tok@ne}{\let\PY@bf=\textbf\def\PY@tc##1{\textcolor[rgb]{0.82,0.25,0.23}{##1}}}
\@namedef{PY@tok@nv}{\def\PY@tc##1{\textcolor[rgb]{0.10,0.09,0.49}{##1}}}
\@namedef{PY@tok@no}{\def\PY@tc##1{\textcolor[rgb]{0.53,0.00,0.00}{##1}}}
\@namedef{PY@tok@nl}{\def\PY@tc##1{\textcolor[rgb]{0.63,0.63,0.00}{##1}}}
\@namedef{PY@tok@ni}{\let\PY@bf=\textbf\def\PY@tc##1{\textcolor[rgb]{0.60,0.60,0.60}{##1}}}
\@namedef{PY@tok@na}{\def\PY@tc##1{\textcolor[rgb]{0.49,0.56,0.16}{##1}}}
\@namedef{PY@tok@nt}{\let\PY@bf=\textbf\def\PY@tc##1{\textcolor[rgb]{0.00,0.50,0.00}{##1}}}
\@namedef{PY@tok@nd}{\def\PY@tc##1{\textcolor[rgb]{0.67,0.13,1.00}{##1}}}
\@namedef{PY@tok@s}{\def\PY@tc##1{\textcolor[rgb]{0.73,0.13,0.13}{##1}}}
\@namedef{PY@tok@sd}{\let\PY@it=\textit\def\PY@tc##1{\textcolor[rgb]{0.73,0.13,0.13}{##1}}}
\@namedef{PY@tok@si}{\let\PY@bf=\textbf\def\PY@tc##1{\textcolor[rgb]{0.73,0.40,0.53}{##1}}}
\@namedef{PY@tok@se}{\let\PY@bf=\textbf\def\PY@tc##1{\textcolor[rgb]{0.73,0.40,0.13}{##1}}}
\@namedef{PY@tok@sr}{\def\PY@tc##1{\textcolor[rgb]{0.73,0.40,0.53}{##1}}}
\@namedef{PY@tok@ss}{\def\PY@tc##1{\textcolor[rgb]{0.10,0.09,0.49}{##1}}}
\@namedef{PY@tok@sx}{\def\PY@tc##1{\textcolor[rgb]{0.00,0.50,0.00}{##1}}}
\@namedef{PY@tok@m}{\def\PY@tc##1{\textcolor[rgb]{0.40,0.40,0.40}{##1}}}
\@namedef{PY@tok@gh}{\let\PY@bf=\textbf\def\PY@tc##1{\textcolor[rgb]{0.00,0.00,0.50}{##1}}}
\@namedef{PY@tok@gu}{\let\PY@bf=\textbf\def\PY@tc##1{\textcolor[rgb]{0.50,0.00,0.50}{##1}}}
\@namedef{PY@tok@gd}{\def\PY@tc##1{\textcolor[rgb]{0.63,0.00,0.00}{##1}}}
\@namedef{PY@tok@gi}{\def\PY@tc##1{\textcolor[rgb]{0.00,0.63,0.00}{##1}}}
\@namedef{PY@tok@gr}{\def\PY@tc##1{\textcolor[rgb]{1.00,0.00,0.00}{##1}}}
\@namedef{PY@tok@ge}{\let\PY@it=\textit}
\@namedef{PY@tok@gs}{\let\PY@bf=\textbf}
\@namedef{PY@tok@gp}{\let\PY@bf=\textbf\def\PY@tc##1{\textcolor[rgb]{0.00,0.00,0.50}{##1}}}
\@namedef{PY@tok@go}{\def\PY@tc##1{\textcolor[rgb]{0.53,0.53,0.53}{##1}}}
\@namedef{PY@tok@gt}{\def\PY@tc##1{\textcolor[rgb]{0.00,0.27,0.87}{##1}}}
\@namedef{PY@tok@err}{\def\PY@bc##1{{\setlength{\fboxsep}{\string -\fboxrule}\fcolorbox[rgb]{1.00,0.00,0.00}{1,1,1}{\strut ##1}}}}
\@namedef{PY@tok@kc}{\let\PY@bf=\textbf\def\PY@tc##1{\textcolor[rgb]{0.00,0.50,0.00}{##1}}}
\@namedef{PY@tok@kd}{\let\PY@bf=\textbf\def\PY@tc##1{\textcolor[rgb]{0.00,0.50,0.00}{##1}}}
\@namedef{PY@tok@kn}{\let\PY@bf=\textbf\def\PY@tc##1{\textcolor[rgb]{0.00,0.50,0.00}{##1}}}
\@namedef{PY@tok@kr}{\let\PY@bf=\textbf\def\PY@tc##1{\textcolor[rgb]{0.00,0.50,0.00}{##1}}}
\@namedef{PY@tok@bp}{\def\PY@tc##1{\textcolor[rgb]{0.00,0.50,0.00}{##1}}}
\@namedef{PY@tok@fm}{\def\PY@tc##1{\textcolor[rgb]{0.00,0.00,1.00}{##1}}}
\@namedef{PY@tok@vc}{\def\PY@tc##1{\textcolor[rgb]{0.10,0.09,0.49}{##1}}}
\@namedef{PY@tok@vg}{\def\PY@tc##1{\textcolor[rgb]{0.10,0.09,0.49}{##1}}}
\@namedef{PY@tok@vi}{\def\PY@tc##1{\textcolor[rgb]{0.10,0.09,0.49}{##1}}}
\@namedef{PY@tok@vm}{\def\PY@tc##1{\textcolor[rgb]{0.10,0.09,0.49}{##1}}}
\@namedef{PY@tok@sa}{\def\PY@tc##1{\textcolor[rgb]{0.73,0.13,0.13}{##1}}}
\@namedef{PY@tok@sb}{\def\PY@tc##1{\textcolor[rgb]{0.73,0.13,0.13}{##1}}}
\@namedef{PY@tok@sc}{\def\PY@tc##1{\textcolor[rgb]{0.73,0.13,0.13}{##1}}}
\@namedef{PY@tok@dl}{\def\PY@tc##1{\textcolor[rgb]{0.73,0.13,0.13}{##1}}}
\@namedef{PY@tok@s2}{\def\PY@tc##1{\textcolor[rgb]{0.73,0.13,0.13}{##1}}}
\@namedef{PY@tok@sh}{\def\PY@tc##1{\textcolor[rgb]{0.73,0.13,0.13}{##1}}}
\@namedef{PY@tok@s1}{\def\PY@tc##1{\textcolor[rgb]{0.73,0.13,0.13}{##1}}}
\@namedef{PY@tok@mb}{\def\PY@tc##1{\textcolor[rgb]{0.40,0.40,0.40}{##1}}}
\@namedef{PY@tok@mf}{\def\PY@tc##1{\textcolor[rgb]{0.40,0.40,0.40}{##1}}}
\@namedef{PY@tok@mh}{\def\PY@tc##1{\textcolor[rgb]{0.40,0.40,0.40}{##1}}}
\@namedef{PY@tok@mi}{\def\PY@tc##1{\textcolor[rgb]{0.40,0.40,0.40}{##1}}}
\@namedef{PY@tok@il}{\def\PY@tc##1{\textcolor[rgb]{0.40,0.40,0.40}{##1}}}
\@namedef{PY@tok@mo}{\def\PY@tc##1{\textcolor[rgb]{0.40,0.40,0.40}{##1}}}
\@namedef{PY@tok@ch}{\let\PY@it=\textit\def\PY@tc##1{\textcolor[rgb]{0.25,0.50,0.50}{##1}}}
\@namedef{PY@tok@cm}{\let\PY@it=\textit\def\PY@tc##1{\textcolor[rgb]{0.25,0.50,0.50}{##1}}}
\@namedef{PY@tok@cpf}{\let\PY@it=\textit\def\PY@tc##1{\textcolor[rgb]{0.25,0.50,0.50}{##1}}}
\@namedef{PY@tok@c1}{\let\PY@it=\textit\def\PY@tc##1{\textcolor[rgb]{0.25,0.50,0.50}{##1}}}
\@namedef{PY@tok@cs}{\let\PY@it=\textit\def\PY@tc##1{\textcolor[rgb]{0.25,0.50,0.50}{##1}}}

\def\PYZbs{\char`\\}
\def\PYZus{\char`\_}
\def\PYZob{\char`\{}
\def\PYZcb{\char`\}}
\def\PYZca{\char`\^}
\def\PYZam{\char`\&}
\def\PYZlt{\char`\<}
\def\PYZgt{\char`\>}
\def\PYZsh{\char`\#}
\def\PYZpc{\char`\%}
\def\PYZdl{\char`\$}
\def\PYZhy{\char`\-}
\def\PYZsq{\char`\'}
\def\PYZdq{\char`\"}
\def\PYZti{\char`\~}
% for compatibility with earlier versions
\def\PYZat{@}
\def\PYZlb{[}
\def\PYZrb{]}
\makeatother


    % For linebreaks inside Verbatim environment from package fancyvrb. 
    \makeatletter
        \newbox\Wrappedcontinuationbox 
        \newbox\Wrappedvisiblespacebox 
        \newcommand*\Wrappedvisiblespace {\textcolor{red}{\textvisiblespace}} 
        \newcommand*\Wrappedcontinuationsymbol {\textcolor{red}{\llap{\tiny$\m@th\hookrightarrow$}}} 
        \newcommand*\Wrappedcontinuationindent {3ex } 
        \newcommand*\Wrappedafterbreak {\kern\Wrappedcontinuationindent\copy\Wrappedcontinuationbox} 
        % Take advantage of the already applied Pygments mark-up to insert 
        % potential linebreaks for TeX processing. 
        %        {, <, #, %, $, ' and ": go to next line. 
        %        _, }, ^, &, >, - and ~: stay at end of broken line. 
        % Use of \textquotesingle for straight quote. 
        \newcommand*\Wrappedbreaksatspecials {% 
            \def\PYGZus{\discretionary{\char`\_}{\Wrappedafterbreak}{\char`\_}}% 
            \def\PYGZob{\discretionary{}{\Wrappedafterbreak\char`\{}{\char`\{}}% 
            \def\PYGZcb{\discretionary{\char`\}}{\Wrappedafterbreak}{\char`\}}}% 
            \def\PYGZca{\discretionary{\char`\^}{\Wrappedafterbreak}{\char`\^}}% 
            \def\PYGZam{\discretionary{\char`\&}{\Wrappedafterbreak}{\char`\&}}% 
            \def\PYGZlt{\discretionary{}{\Wrappedafterbreak\char`\<}{\char`\<}}% 
            \def\PYGZgt{\discretionary{\char`\>}{\Wrappedafterbreak}{\char`\>}}% 
            \def\PYGZsh{\discretionary{}{\Wrappedafterbreak\char`\#}{\char`\#}}% 
            \def\PYGZpc{\discretionary{}{\Wrappedafterbreak\char`\%}{\char`\%}}% 
            \def\PYGZdl{\discretionary{}{\Wrappedafterbreak\char`\$}{\char`\$}}% 
            \def\PYGZhy{\discretionary{\char`\-}{\Wrappedafterbreak}{\char`\-}}% 
            \def\PYGZsq{\discretionary{}{\Wrappedafterbreak\textquotesingle}{\textquotesingle}}% 
            \def\PYGZdq{\discretionary{}{\Wrappedafterbreak\char`\"}{\char`\"}}% 
            \def\PYGZti{\discretionary{\char`\~}{\Wrappedafterbreak}{\char`\~}}% 
        } 
        % Some characters . , ; ? ! / are not pygmentized. 
        % This macro makes them "active" and they will insert potential linebreaks 
        \newcommand*\Wrappedbreaksatpunct {% 
            \lccode`\~`\.\lowercase{\def~}{\discretionary{\hbox{\char`\.}}{\Wrappedafterbreak}{\hbox{\char`\.}}}% 
            \lccode`\~`\,\lowercase{\def~}{\discretionary{\hbox{\char`\,}}{\Wrappedafterbreak}{\hbox{\char`\,}}}% 
            \lccode`\~`\;\lowercase{\def~}{\discretionary{\hbox{\char`\;}}{\Wrappedafterbreak}{\hbox{\char`\;}}}% 
            \lccode`\~`\:\lowercase{\def~}{\discretionary{\hbox{\char`\:}}{\Wrappedafterbreak}{\hbox{\char`\:}}}% 
            \lccode`\~`\?\lowercase{\def~}{\discretionary{\hbox{\char`\?}}{\Wrappedafterbreak}{\hbox{\char`\?}}}% 
            \lccode`\~`\!\lowercase{\def~}{\discretionary{\hbox{\char`\!}}{\Wrappedafterbreak}{\hbox{\char`\!}}}% 
            \lccode`\~`\/\lowercase{\def~}{\discretionary{\hbox{\char`\/}}{\Wrappedafterbreak}{\hbox{\char`\/}}}% 
            \catcode`\.\active
            \catcode`\,\active 
            \catcode`\;\active
            \catcode`\:\active
            \catcode`\?\active
            \catcode`\!\active
            \catcode`\/\active 
            \lccode`\~`\~ 	
        }
    \makeatother

    \let\OriginalVerbatim=\Verbatim
    \makeatletter
    \renewcommand{\Verbatim}[1][1]{%
        %\parskip\z@skip
        \sbox\Wrappedcontinuationbox {\Wrappedcontinuationsymbol}%
        \sbox\Wrappedvisiblespacebox {\FV@SetupFont\Wrappedvisiblespace}%
        \def\FancyVerbFormatLine ##1{\hsize\linewidth
            \vtop{\raggedright\hyphenpenalty\z@\exhyphenpenalty\z@
                \doublehyphendemerits\z@\finalhyphendemerits\z@
                \strut ##1\strut}%
        }%
        % If the linebreak is at a space, the latter will be displayed as visible
        % space at end of first line, and a continuation symbol starts next line.
        % Stretch/shrink are however usually zero for typewriter font.
        \def\FV@Space {%
            \nobreak\hskip\z@ plus\fontdimen3\font minus\fontdimen4\font
            \discretionary{\copy\Wrappedvisiblespacebox}{\Wrappedafterbreak}
            {\kern\fontdimen2\font}%
        }%
        
        % Allow breaks at special characters using \PYG... macros.
        \Wrappedbreaksatspecials
        % Breaks at punctuation characters . , ; ? ! and / need catcode=\active 	
        \OriginalVerbatim[#1,codes*=\Wrappedbreaksatpunct]%
    }
    \makeatother

    % Exact colors from NB
    \definecolor{incolor}{HTML}{303F9F}
    \definecolor{outcolor}{HTML}{D84315}
    \definecolor{cellborder}{HTML}{CFCFCF}
    \definecolor{cellbackground}{HTML}{F7F7F7}
    
    % prompt
    \makeatletter
    \newcommand{\boxspacing}{\kern\kvtcb@left@rule\kern\kvtcb@boxsep}
    \makeatother
    \newcommand{\prompt}[4]{
        {\ttfamily\llap{{\color{#2}[#3]:\hspace{3pt}#4}}\vspace{-\baselineskip}}
    }
    

    
    % Prevent overflowing lines due to hard-to-break entities
    \sloppy 
    % Setup hyperref package
    \hypersetup{
      breaklinks=true,  % so long urls are correctly broken across lines
      colorlinks=true,
      urlcolor=urlcolor,
      linkcolor=linkcolor,
      citecolor=citecolor,
      }
    % Slightly bigger margins than the latex defaults
    
    \geometry{verbose,tmargin=1in,bmargin=1in,lmargin=1in,rmargin=1in}
    
    

\begin{document}
    
    \maketitle
    
    

    
    \hypertarget{control-of-weir-flow-by-changing-geometry}{%
\section{Control of Weir Flow by Changing
Geometry}\label{control-of-weir-flow-by-changing-geometry}}

\hypertarget{problem}{%
\subsection{Problem}\label{problem}}

\hypertarget{statement}{%
\subsubsection{Statement}\label{statement}}

We will try and understand how height of fluid over weir affects the
flow rate of water across the weir. We will analyse 2 cases for each
height. One with laminar flow and another with turbulent.

\hypertarget{properties}{%
\subsubsection{Properties}\label{properties}}

\begin{itemize}
\tightlist
\item
  Fixed:

  \begin{enumerate}
  \def\labelenumi{\arabic{enumi}.}
  \tightlist
  \item
    \(\text{Dimensions of mesh}\ (l = 25,\ w = 6)\)
  \item
    \(\text{Width of weir (L)} = 1\ m\)
  \item
    \(\text{Distance of weir from inlet} = 15\ m\)
  \end{enumerate}
\item
  Variables:

  \begin{enumerate}
  \def\labelenumi{\arabic{enumi}.}
  \tightlist
  \item
    \(H_w = \text{Height of weir}\)
  \item
    \(H_s = \text{Height of the stream}\)
  \item
    \(H = H_s - H_w = \text{Height of the stream above the weir's crest}\)
  \end{enumerate}
\end{itemize}

\hypertarget{geometry}{%
\subsubsection{Geometry:}\label{geometry}}

\includegraphics{C:/Users/lolop/Repositories/TeXURL/test/.TeXURL_dump/b3440b20.png} \textbf{Figure 1.1:}
3D geometry of flow over weir

\includegraphics{C:/Users/lolop/Repositories/TeXURL/test/.TeXURL_dump/af1b0aab.png} \textbf{Figure 1.2:}
Cross section view

\includegraphics{C:/Users/lolop/Repositories/TeXURL/test/.TeXURL_dump/af590ab0.png} \textbf{Figure 1.3:}
Mesh cross section view

\hypertarget{boundary-conditions}{%
\subsubsection{Boundary conditions:}\label{boundary-conditions}}

\begin{enumerate}
\def\labelenumi{\arabic{enumi}.}
\tightlist
\item
  No slip condition on the stream bed \& the surface of weir.
\item
  Plug flow in the inlet stream.
\end{enumerate}

\hypertarget{solving}{%
\subsubsection{Solving}\label{solving}}

\begin{itemize}
\tightlist
\item
  K-epsilon turbulence model (2 equations) is used.
\item
  Divisions per unit length in the mesh = 2.
\item
  Volume ratio used is modified high resolution schemes for interface
  capturing (HRIC).
\end{itemize}

\hypertarget{geometry---gmsh-file}{%
\subsection{\texorpdfstring{Geometry -
\texttt{Gmsh\ file}}{Geometry - Gmsh file}}\label{geometry---gmsh-file}}

\hypertarget{case-1-h-0-m}{%
\subsubsection{\texorpdfstring{Case 1:
\(H = 0\ m\)}{Case 1: H = 0\textbackslash{} m}}\label{case-1-h-0-m}}

\hypertarget{properties-1}{%
\paragraph{Properties}\label{properties-1}}

\(H_s = 3\ m, H_w = 3\ m\)

\includegraphics{C:/Users/lolop/Repositories/TeXURL/test/.TeXURL_dump/b0170ad3.jpg} \textbf{Figure 2.1:}
2D mesh for case 1 (H=0)

\hypertarget{mesh-file}{%
\paragraph{Mesh file}\label{mesh-file}}

\begin{Shaded}
\begin{Highlighting}[]
\CommentTok{// Gmsh project created on Mon May 10 10:15:28 2021}
\NormalTok{SetFactory}\OperatorTok{(}\StringTok{"OpenCASCADE"}\OperatorTok{);}
\NormalTok{Point}\OperatorTok{(}\DecValTok{1}\OperatorTok{)} \OperatorTok{=} \OperatorTok{\{}\DecValTok{0}\OperatorTok{,} \DecValTok{0}\OperatorTok{,} \DecValTok{0}\OperatorTok{,} \FloatTok{1.0}\OperatorTok{\};}
\NormalTok{Point}\OperatorTok{(}\DecValTok{2}\OperatorTok{)} \OperatorTok{=} \OperatorTok{\{}\DecValTok{15}\OperatorTok{,} \DecValTok{0}\OperatorTok{,} \DecValTok{0}\OperatorTok{,} \FloatTok{1.0}\OperatorTok{\};}
\NormalTok{Point}\OperatorTok{(}\DecValTok{3}\OperatorTok{)} \OperatorTok{=} \OperatorTok{\{}\DecValTok{15}\OperatorTok{,} \DecValTok{3}\OperatorTok{,} \DecValTok{0}\OperatorTok{,} \FloatTok{1.0}\OperatorTok{\};}
\NormalTok{Point}\OperatorTok{(}\DecValTok{4}\OperatorTok{)} \OperatorTok{=} \OperatorTok{\{}\DecValTok{16}\OperatorTok{,} \DecValTok{3}\OperatorTok{,} \DecValTok{0}\OperatorTok{,} \FloatTok{1.0}\OperatorTok{\};}
\NormalTok{Point}\OperatorTok{(}\DecValTok{5}\OperatorTok{)} \OperatorTok{=} \OperatorTok{\{}\DecValTok{16}\OperatorTok{,} \DecValTok{0}\OperatorTok{,} \DecValTok{0}\OperatorTok{,} \FloatTok{1.0}\OperatorTok{\};}
\NormalTok{Point}\OperatorTok{(}\DecValTok{6}\OperatorTok{)} \OperatorTok{=} \OperatorTok{\{}\DecValTok{25}\OperatorTok{,} \DecValTok{0}\OperatorTok{,} \DecValTok{0}\OperatorTok{,} \FloatTok{1.0}\OperatorTok{\};}
\NormalTok{Point}\OperatorTok{(}\DecValTok{7}\OperatorTok{)} \OperatorTok{=} \OperatorTok{\{}\DecValTok{25}\OperatorTok{,} \DecValTok{6}\OperatorTok{,} \DecValTok{0}\OperatorTok{,} \FloatTok{1.0}\OperatorTok{\};}
\NormalTok{Point}\OperatorTok{(}\DecValTok{8}\OperatorTok{)} \OperatorTok{=} \OperatorTok{\{}\DecValTok{0}\OperatorTok{,} \DecValTok{6}\OperatorTok{,} \DecValTok{0}\OperatorTok{,} \FloatTok{1.0}\OperatorTok{\};}
\NormalTok{Point}\OperatorTok{(}\DecValTok{9}\OperatorTok{)} \OperatorTok{=} \OperatorTok{\{}\DecValTok{0}\OperatorTok{,} \DecValTok{3}\OperatorTok{,} \DecValTok{0}\OperatorTok{,} \FloatTok{1.0}\OperatorTok{\};}
\NormalTok{Line}\OperatorTok{(}\DecValTok{1}\OperatorTok{)} \OperatorTok{=} \OperatorTok{\{}\DecValTok{1}\OperatorTok{,} \DecValTok{2}\OperatorTok{\};}
\NormalTok{Line}\OperatorTok{(}\DecValTok{2}\OperatorTok{)} \OperatorTok{=} \OperatorTok{\{}\DecValTok{2}\OperatorTok{,} \DecValTok{3}\OperatorTok{\};}
\NormalTok{Line}\OperatorTok{(}\DecValTok{3}\OperatorTok{)} \OperatorTok{=} \OperatorTok{\{}\DecValTok{3}\OperatorTok{,} \DecValTok{4}\OperatorTok{\};}
\NormalTok{Line}\OperatorTok{(}\DecValTok{4}\OperatorTok{)} \OperatorTok{=} \OperatorTok{\{}\DecValTok{4}\OperatorTok{,} \DecValTok{5}\OperatorTok{\};}
\NormalTok{Line}\OperatorTok{(}\DecValTok{5}\OperatorTok{)} \OperatorTok{=} \OperatorTok{\{}\DecValTok{5}\OperatorTok{,} \DecValTok{6}\OperatorTok{\};}
\NormalTok{Line}\OperatorTok{(}\DecValTok{6}\OperatorTok{)} \OperatorTok{=} \OperatorTok{\{}\DecValTok{6}\OperatorTok{,} \DecValTok{7}\OperatorTok{\};}
\NormalTok{Line}\OperatorTok{(}\DecValTok{7}\OperatorTok{)} \OperatorTok{=} \OperatorTok{\{}\DecValTok{7}\OperatorTok{,} \DecValTok{8}\OperatorTok{\};}
\NormalTok{Line}\OperatorTok{(}\DecValTok{8}\OperatorTok{)} \OperatorTok{=} \OperatorTok{\{}\DecValTok{8}\OperatorTok{,} \DecValTok{9}\OperatorTok{\};}
\NormalTok{Line}\OperatorTok{(}\DecValTok{9}\OperatorTok{)} \OperatorTok{=} \OperatorTok{\{}\DecValTok{9}\OperatorTok{,} \DecValTok{1}\OperatorTok{\};}
\NormalTok{Curve Loop}\OperatorTok{(}\DecValTok{1}\OperatorTok{)} \OperatorTok{=} \OperatorTok{\{}\DecValTok{8}\OperatorTok{,} \DecValTok{9}\OperatorTok{,} \DecValTok{1}\OperatorTok{,} \DecValTok{2}\OperatorTok{,} \DecValTok{3}\OperatorTok{,} \DecValTok{4}\OperatorTok{,} \DecValTok{5}\OperatorTok{,} \DecValTok{6}\OperatorTok{,} \DecValTok{7}\OperatorTok{\};}
\NormalTok{Plane Surface}\OperatorTok{(}\DecValTok{1}\OperatorTok{)} \OperatorTok{=} \OperatorTok{\{}\DecValTok{1}\OperatorTok{\};}
\NormalTok{Physical Curve}\OperatorTok{(}\StringTok{"inlet"}\OperatorTok{,} \DecValTok{10}\OperatorTok{)} \OperatorTok{=} \OperatorTok{\{}\DecValTok{9}\OperatorTok{\};}
\NormalTok{Physical Curve}\OperatorTok{(}\StringTok{"outlet"}\OperatorTok{,} \DecValTok{11}\OperatorTok{)} \OperatorTok{=} \OperatorTok{\{}\DecValTok{6}\OperatorTok{,} \DecValTok{5}\OperatorTok{\};}
\NormalTok{Physical Curve}\OperatorTok{(}\StringTok{"ambient"}\OperatorTok{,} \DecValTok{12}\OperatorTok{)} \OperatorTok{=} \OperatorTok{\{}\DecValTok{7}\OperatorTok{\};}
\NormalTok{Physical Curve}\OperatorTok{(}\StringTok{"fw"}\OperatorTok{,} \DecValTok{13}\OperatorTok{)} \OperatorTok{=} \OperatorTok{\{}\DecValTok{8}\OperatorTok{,} \DecValTok{1}\OperatorTok{,} \DecValTok{2}\OperatorTok{,} \DecValTok{3}\OperatorTok{,} \DecValTok{4}\OperatorTok{\};}
\NormalTok{Physical Surface}\OperatorTok{(}\StringTok{"Area1"}\OperatorTok{,} \DecValTok{14}\OperatorTok{)} \OperatorTok{=} \OperatorTok{\{}\DecValTok{1}\OperatorTok{\};}
\NormalTok{Transfinite Curve }\OperatorTok{\{}\DecValTok{7}\OperatorTok{\}} \OperatorTok{=} \DecValTok{50}\NormalTok{ Using Progression }\DecValTok{1}\OperatorTok{;}
\NormalTok{Transfinite Curve }\OperatorTok{\{}\DecValTok{6}\OperatorTok{\}} \OperatorTok{=} \DecValTok{12}\NormalTok{ Using Progression }\DecValTok{1}\OperatorTok{;}
\NormalTok{Transfinite Curve }\OperatorTok{\{}\DecValTok{8}\OperatorTok{\}} \OperatorTok{=} \DecValTok{6}\NormalTok{ Using Progression }\DecValTok{1}\OperatorTok{;}
\NormalTok{Transfinite Curve }\OperatorTok{\{}\DecValTok{9}\OperatorTok{\}} \OperatorTok{=} \DecValTok{6}\NormalTok{ Using Progression }\DecValTok{1}\OperatorTok{;}
\NormalTok{Transfinite Curve }\OperatorTok{\{}\DecValTok{1}\OperatorTok{\}} \OperatorTok{=} \DecValTok{30}\NormalTok{ Using Progression }\DecValTok{1}\OperatorTok{;}
\NormalTok{Transfinite Curve }\OperatorTok{\{}\DecValTok{5}\OperatorTok{\}} \OperatorTok{=} \DecValTok{18}\NormalTok{ Using Progression }\DecValTok{1}\OperatorTok{;}
\NormalTok{Transfinite Curve }\OperatorTok{\{}\DecValTok{2}\OperatorTok{\}} \OperatorTok{=} \DecValTok{6}\NormalTok{ Using Progression }\DecValTok{1}\OperatorTok{;}
\NormalTok{Transfinite Curve }\OperatorTok{\{}\DecValTok{4}\OperatorTok{\}} \OperatorTok{=} \DecValTok{6}\NormalTok{ Using Progression }\DecValTok{1}\OperatorTok{;}
\NormalTok{Transfinite Curve }\OperatorTok{\{}\DecValTok{3}\OperatorTok{\}} \OperatorTok{=} \DecValTok{2}\NormalTok{ Using Progression }\DecValTok{1}\OperatorTok{;}
\end{Highlighting}
\end{Shaded}

\hypertarget{case-2-h-0.5-m}{%
\subsubsection{\texorpdfstring{Case 2:
\(H = 0.5\ m\)}{Case 2: H = 0.5\textbackslash{} m}}\label{case-2-h-0.5-m}}

\hypertarget{properties-2}{%
\paragraph{Properties}\label{properties-2}}

\(H_s = 3\ m, H_w = 2.5\ m\)

\includegraphics{C:/Users/lolop/Repositories/TeXURL/test/.TeXURL_dump/b2bc0b1b.jpg} \textbf{Figure 2.2:}
2D mesh for case 2 (H=0.5)

\hypertarget{mesh-file-1}{%
\paragraph{Mesh file}\label{mesh-file-1}}

\begin{Shaded}
\begin{Highlighting}[]
\NormalTok{SetFactory}\OperatorTok{(}\StringTok{"OpenCASCADE"}\OperatorTok{);}
\NormalTok{Point}\OperatorTok{(}\DecValTok{1}\OperatorTok{)} \OperatorTok{=} \OperatorTok{\{}\DecValTok{0}\OperatorTok{,} \DecValTok{0}\OperatorTok{,} \DecValTok{0}\OperatorTok{,} \FloatTok{1.0}\OperatorTok{\};}
\NormalTok{Point}\OperatorTok{(}\DecValTok{2}\OperatorTok{)} \OperatorTok{=} \OperatorTok{\{}\DecValTok{15}\OperatorTok{,} \DecValTok{0}\OperatorTok{,} \DecValTok{0}\OperatorTok{,} \FloatTok{1.0}\OperatorTok{\};}
\NormalTok{Point}\OperatorTok{(}\DecValTok{3}\OperatorTok{)} \OperatorTok{=} \OperatorTok{\{}\DecValTok{15}\OperatorTok{,} \FloatTok{2.5}\OperatorTok{,} \DecValTok{0}\OperatorTok{,} \FloatTok{1.0}\OperatorTok{\};}
\NormalTok{Point}\OperatorTok{(}\DecValTok{4}\OperatorTok{)} \OperatorTok{=} \OperatorTok{\{}\DecValTok{16}\OperatorTok{,} \FloatTok{2.5}\OperatorTok{,} \DecValTok{0}\OperatorTok{,} \FloatTok{1.0}\OperatorTok{\};}
\NormalTok{Point}\OperatorTok{(}\DecValTok{5}\OperatorTok{)} \OperatorTok{=} \OperatorTok{\{}\DecValTok{16}\OperatorTok{,} \DecValTok{0}\OperatorTok{,} \DecValTok{0}\OperatorTok{,} \FloatTok{1.0}\OperatorTok{\};}
\NormalTok{Point}\OperatorTok{(}\DecValTok{6}\OperatorTok{)} \OperatorTok{=} \OperatorTok{\{}\DecValTok{25}\OperatorTok{,} \DecValTok{0}\OperatorTok{,} \DecValTok{0}\OperatorTok{,} \FloatTok{1.0}\OperatorTok{\};}
\NormalTok{Point}\OperatorTok{(}\DecValTok{7}\OperatorTok{)} \OperatorTok{=} \OperatorTok{\{}\DecValTok{25}\OperatorTok{,} \DecValTok{6}\OperatorTok{,} \DecValTok{0}\OperatorTok{,} \FloatTok{1.0}\OperatorTok{\};}
\NormalTok{Point}\OperatorTok{(}\DecValTok{8}\OperatorTok{)} \OperatorTok{=} \OperatorTok{\{}\DecValTok{0}\OperatorTok{,} \DecValTok{6}\OperatorTok{,} \DecValTok{0}\OperatorTok{,} \FloatTok{1.0}\OperatorTok{\};}
\NormalTok{Point}\OperatorTok{(}\DecValTok{9}\OperatorTok{)} \OperatorTok{=} \OperatorTok{\{}\DecValTok{0}\OperatorTok{,} \DecValTok{3}\OperatorTok{,} \DecValTok{0}\OperatorTok{,} \FloatTok{1.0}\OperatorTok{\};}
\NormalTok{Line}\OperatorTok{(}\DecValTok{1}\OperatorTok{)} \OperatorTok{=} \OperatorTok{\{}\DecValTok{1}\OperatorTok{,} \DecValTok{2}\OperatorTok{\};}
\NormalTok{Line}\OperatorTok{(}\DecValTok{2}\OperatorTok{)} \OperatorTok{=} \OperatorTok{\{}\DecValTok{2}\OperatorTok{,} \DecValTok{3}\OperatorTok{\};}
\NormalTok{Line}\OperatorTok{(}\DecValTok{3}\OperatorTok{)} \OperatorTok{=} \OperatorTok{\{}\DecValTok{3}\OperatorTok{,} \DecValTok{4}\OperatorTok{\};}
\NormalTok{Line}\OperatorTok{(}\DecValTok{4}\OperatorTok{)} \OperatorTok{=} \OperatorTok{\{}\DecValTok{4}\OperatorTok{,} \DecValTok{5}\OperatorTok{\};}
\NormalTok{Line}\OperatorTok{(}\DecValTok{5}\OperatorTok{)} \OperatorTok{=} \OperatorTok{\{}\DecValTok{5}\OperatorTok{,} \DecValTok{6}\OperatorTok{\};}
\NormalTok{Line}\OperatorTok{(}\DecValTok{6}\OperatorTok{)} \OperatorTok{=} \OperatorTok{\{}\DecValTok{6}\OperatorTok{,} \DecValTok{7}\OperatorTok{\};}
\NormalTok{Line}\OperatorTok{(}\DecValTok{7}\OperatorTok{)} \OperatorTok{=} \OperatorTok{\{}\DecValTok{7}\OperatorTok{,} \DecValTok{8}\OperatorTok{\};}
\NormalTok{Line}\OperatorTok{(}\DecValTok{8}\OperatorTok{)} \OperatorTok{=} \OperatorTok{\{}\DecValTok{8}\OperatorTok{,} \DecValTok{9}\OperatorTok{\};}
\NormalTok{Line}\OperatorTok{(}\DecValTok{9}\OperatorTok{)} \OperatorTok{=} \OperatorTok{\{}\DecValTok{9}\OperatorTok{,} \DecValTok{1}\OperatorTok{\};}
\NormalTok{Curve Loop}\OperatorTok{(}\DecValTok{1}\OperatorTok{)} \OperatorTok{=} \OperatorTok{\{}\DecValTok{9}\OperatorTok{,} \DecValTok{1}\OperatorTok{,} \DecValTok{2}\OperatorTok{,} \DecValTok{3}\OperatorTok{,} \DecValTok{4}\OperatorTok{,} \DecValTok{5}\OperatorTok{,} \DecValTok{6}\OperatorTok{,} \DecValTok{7}\OperatorTok{,} \DecValTok{8}\OperatorTok{\};}
\NormalTok{Plane Surface}\OperatorTok{(}\DecValTok{1}\OperatorTok{)} \OperatorTok{=} \OperatorTok{\{}\DecValTok{1}\OperatorTok{\};}
\NormalTok{Physical Curve}\OperatorTok{(}\StringTok{"inlet"}\OperatorTok{,} \DecValTok{10}\OperatorTok{)} \OperatorTok{=} \OperatorTok{\{}\DecValTok{9}\OperatorTok{\};}
\NormalTok{Physical Curve}\OperatorTok{(}\StringTok{"outlet"}\OperatorTok{,} \DecValTok{11}\OperatorTok{)} \OperatorTok{=} \OperatorTok{\{}\DecValTok{6}\OperatorTok{,} \DecValTok{5}\OperatorTok{\};}
\NormalTok{Physical Curve}\OperatorTok{(}\StringTok{"ambient"}\OperatorTok{,} \DecValTok{12}\OperatorTok{)} \OperatorTok{=} \OperatorTok{\{}\DecValTok{7}\OperatorTok{\};}
\NormalTok{Physical Curve}\OperatorTok{(}\StringTok{"fw"}\OperatorTok{,} \DecValTok{13}\OperatorTok{)} \OperatorTok{=} \OperatorTok{\{}\DecValTok{8}\OperatorTok{,} \DecValTok{1}\OperatorTok{,} \DecValTok{2}\OperatorTok{,} \DecValTok{3}\OperatorTok{,} \DecValTok{4}\OperatorTok{\};}
\NormalTok{Physical Surface}\OperatorTok{(}\StringTok{"Area2"}\OperatorTok{,} \DecValTok{14}\OperatorTok{)} \OperatorTok{=} \OperatorTok{\{}\DecValTok{1}\OperatorTok{\};}
\NormalTok{SetFactory}\OperatorTok{(}\StringTok{"OpenCASCADE"}\OperatorTok{);}
\NormalTok{Transfinite Curve }\OperatorTok{\{}\DecValTok{7}\OperatorTok{\}} \OperatorTok{=} \DecValTok{50}\NormalTok{ Using Progression }\DecValTok{1}\OperatorTok{;}
\NormalTok{Transfinite Curve }\OperatorTok{\{}\DecValTok{6}\OperatorTok{\}} \OperatorTok{=} \DecValTok{12}\NormalTok{ Using Progression }\DecValTok{1}\OperatorTok{;}
\NormalTok{Transfinite Curve }\OperatorTok{\{}\DecValTok{8}\OperatorTok{\}} \OperatorTok{=} \DecValTok{6}\NormalTok{ Using Progression }\DecValTok{1}\OperatorTok{;}
\NormalTok{Transfinite Curve }\OperatorTok{\{}\DecValTok{9}\OperatorTok{\}} \OperatorTok{=} \DecValTok{6}\NormalTok{ Using Progression }\DecValTok{1}\OperatorTok{;}
\NormalTok{Transfinite Curve }\OperatorTok{\{}\DecValTok{1}\OperatorTok{\}} \OperatorTok{=} \DecValTok{30}\NormalTok{ Using Progression }\DecValTok{1}\OperatorTok{;}
\NormalTok{Transfinite Curve }\OperatorTok{\{}\DecValTok{5}\OperatorTok{\}} \OperatorTok{=} \DecValTok{18}\NormalTok{ Using Progression }\DecValTok{1}\OperatorTok{;}
\NormalTok{Transfinite Curve }\OperatorTok{\{}\DecValTok{2}\OperatorTok{\}} \OperatorTok{=} \DecValTok{5}\NormalTok{ Using Progression }\DecValTok{1}\OperatorTok{;}
\NormalTok{Transfinite Curve }\OperatorTok{\{}\DecValTok{4}\OperatorTok{\}} \OperatorTok{=} \DecValTok{5}\NormalTok{ Using Progression }\DecValTok{1}\OperatorTok{;}
\NormalTok{Transfinite Curve }\OperatorTok{\{}\DecValTok{3}\OperatorTok{\}} \OperatorTok{=} \DecValTok{2}\NormalTok{ Using Progression }\DecValTok{1}\OperatorTok{;}
\end{Highlighting}
\end{Shaded}

\hypertarget{case-3-h-1-m}{%
\subsubsection{\texorpdfstring{Case 3:
\(H = 1\ m\)}{Case 3: H = 1\textbackslash{} m}}\label{case-3-h-1-m}}

\hypertarget{properties-3}{%
\paragraph{Properties}\label{properties-3}}

\(H_w = 2\ m, H_s = 3\ m\)

\includegraphics{C:/Users/lolop/Repositories/TeXURL/test/.TeXURL_dump/afff0abd.jpg} \textbf{Figure 2.3:}
2D mesh for case 3 (H=1)

\hypertarget{mesh-file-2}{%
\paragraph{Mesh file}\label{mesh-file-2}}

\begin{Shaded}
\begin{Highlighting}[]
\NormalTok{SetFactory}\OperatorTok{(}\StringTok{"OpenCASCADE"}\OperatorTok{);}
\NormalTok{Point}\OperatorTok{(}\DecValTok{1}\OperatorTok{)} \OperatorTok{=} \OperatorTok{\{}\DecValTok{0}\OperatorTok{,} \DecValTok{0}\OperatorTok{,} \DecValTok{0}\OperatorTok{,} \FloatTok{1.0}\OperatorTok{\};}
\NormalTok{Point}\OperatorTok{(}\DecValTok{2}\OperatorTok{)} \OperatorTok{=} \OperatorTok{\{}\DecValTok{15}\OperatorTok{,} \DecValTok{0}\OperatorTok{,} \DecValTok{0}\OperatorTok{,} \FloatTok{1.0}\OperatorTok{\};}
\NormalTok{Point}\OperatorTok{(}\DecValTok{3}\OperatorTok{)} \OperatorTok{=} \OperatorTok{\{}\DecValTok{15}\OperatorTok{,} \DecValTok{2}\OperatorTok{,} \DecValTok{0}\OperatorTok{,} \FloatTok{1.0}\OperatorTok{\};}
\NormalTok{Point}\OperatorTok{(}\DecValTok{4}\OperatorTok{)} \OperatorTok{=} \OperatorTok{\{}\DecValTok{16}\OperatorTok{,} \DecValTok{2}\OperatorTok{,} \DecValTok{0}\OperatorTok{,} \FloatTok{1.0}\OperatorTok{\};}
\NormalTok{Point}\OperatorTok{(}\DecValTok{5}\OperatorTok{)} \OperatorTok{=} \OperatorTok{\{}\DecValTok{16}\OperatorTok{,} \DecValTok{0}\OperatorTok{,} \DecValTok{0}\OperatorTok{,} \FloatTok{1.0}\OperatorTok{\};}
\NormalTok{Point}\OperatorTok{(}\DecValTok{6}\OperatorTok{)} \OperatorTok{=} \OperatorTok{\{}\DecValTok{25}\OperatorTok{,} \DecValTok{0}\OperatorTok{,} \DecValTok{0}\OperatorTok{,} \FloatTok{1.0}\OperatorTok{\};}
\NormalTok{Point}\OperatorTok{(}\DecValTok{7}\OperatorTok{)} \OperatorTok{=} \OperatorTok{\{}\DecValTok{25}\OperatorTok{,} \DecValTok{6}\OperatorTok{,} \DecValTok{0}\OperatorTok{,} \FloatTok{1.0}\OperatorTok{\};}
\NormalTok{Point}\OperatorTok{(}\DecValTok{8}\OperatorTok{)} \OperatorTok{=} \OperatorTok{\{}\DecValTok{0}\OperatorTok{,} \DecValTok{6}\OperatorTok{,} \DecValTok{0}\OperatorTok{,} \FloatTok{1.0}\OperatorTok{\};}
\NormalTok{Point}\OperatorTok{(}\DecValTok{9}\OperatorTok{)} \OperatorTok{=} \OperatorTok{\{}\DecValTok{0}\OperatorTok{,} \DecValTok{3}\OperatorTok{,} \DecValTok{0}\OperatorTok{,} \FloatTok{1.0}\OperatorTok{\};}
\NormalTok{Line}\OperatorTok{(}\DecValTok{1}\OperatorTok{)} \OperatorTok{=} \OperatorTok{\{}\DecValTok{1}\OperatorTok{,} \DecValTok{2}\OperatorTok{\};}
\NormalTok{Line}\OperatorTok{(}\DecValTok{2}\OperatorTok{)} \OperatorTok{=} \OperatorTok{\{}\DecValTok{2}\OperatorTok{,} \DecValTok{3}\OperatorTok{\};}
\NormalTok{Line}\OperatorTok{(}\DecValTok{3}\OperatorTok{)} \OperatorTok{=} \OperatorTok{\{}\DecValTok{3}\OperatorTok{,} \DecValTok{4}\OperatorTok{\};}
\NormalTok{Line}\OperatorTok{(}\DecValTok{4}\OperatorTok{)} \OperatorTok{=} \OperatorTok{\{}\DecValTok{4}\OperatorTok{,} \DecValTok{5}\OperatorTok{\};}
\NormalTok{Line}\OperatorTok{(}\DecValTok{5}\OperatorTok{)} \OperatorTok{=} \OperatorTok{\{}\DecValTok{5}\OperatorTok{,} \DecValTok{6}\OperatorTok{\};}
\NormalTok{Line}\OperatorTok{(}\DecValTok{6}\OperatorTok{)} \OperatorTok{=} \OperatorTok{\{}\DecValTok{6}\OperatorTok{,} \DecValTok{7}\OperatorTok{\};}
\NormalTok{Line}\OperatorTok{(}\DecValTok{7}\OperatorTok{)} \OperatorTok{=} \OperatorTok{\{}\DecValTok{7}\OperatorTok{,} \DecValTok{8}\OperatorTok{\};}
\NormalTok{Line}\OperatorTok{(}\DecValTok{8}\OperatorTok{)} \OperatorTok{=} \OperatorTok{\{}\DecValTok{8}\OperatorTok{,} \DecValTok{9}\OperatorTok{\};}
\NormalTok{Line}\OperatorTok{(}\DecValTok{9}\OperatorTok{)} \OperatorTok{=} \OperatorTok{\{}\DecValTok{9}\OperatorTok{,} \DecValTok{1}\OperatorTok{\};}
\NormalTok{Curve Loop}\OperatorTok{(}\DecValTok{1}\OperatorTok{)} \OperatorTok{=} \OperatorTok{\{}\DecValTok{8}\OperatorTok{,} \DecValTok{9}\OperatorTok{,} \DecValTok{1}\OperatorTok{,} \DecValTok{2}\OperatorTok{,} \DecValTok{3}\OperatorTok{,} \DecValTok{4}\OperatorTok{,} \DecValTok{5}\OperatorTok{,} \DecValTok{6}\OperatorTok{,} \DecValTok{7}\OperatorTok{\};}
\NormalTok{Plane Surface}\OperatorTok{(}\DecValTok{1}\OperatorTok{)} \OperatorTok{=} \OperatorTok{\{}\DecValTok{1}\OperatorTok{\};}
\NormalTok{Physical Curve}\OperatorTok{(}\StringTok{"inlet"}\OperatorTok{,} \DecValTok{10}\OperatorTok{)} \OperatorTok{=} \OperatorTok{\{}\DecValTok{9}\OperatorTok{\};}
\NormalTok{Physical Curve}\OperatorTok{(}\StringTok{"outlet"}\OperatorTok{,} \DecValTok{11}\OperatorTok{)} \OperatorTok{=} \OperatorTok{\{}\DecValTok{6}\OperatorTok{,} \DecValTok{5}\OperatorTok{\};}
\NormalTok{Physical Curve}\OperatorTok{(}\StringTok{"ambient"}\OperatorTok{,} \DecValTok{12}\OperatorTok{)} \OperatorTok{=} \OperatorTok{\{}\DecValTok{7}\OperatorTok{\};}
\NormalTok{Physical Curve}\OperatorTok{(}\StringTok{"fw"}\OperatorTok{,} \DecValTok{13}\OperatorTok{)} \OperatorTok{=} \OperatorTok{\{}\DecValTok{8}\OperatorTok{,} \DecValTok{1}\OperatorTok{,} \DecValTok{2}\OperatorTok{,} \DecValTok{3}\OperatorTok{,} \DecValTok{4}\OperatorTok{\};}
\NormalTok{Physical Surface}\OperatorTok{(}\StringTok{"Area3"}\OperatorTok{,} \DecValTok{14}\OperatorTok{)} \OperatorTok{=} \OperatorTok{\{}\DecValTok{1}\OperatorTok{\};}
\NormalTok{Transfinite Curve }\OperatorTok{\{}\DecValTok{7}\OperatorTok{\}} \OperatorTok{=} \DecValTok{50}\NormalTok{ Using Progression }\DecValTok{1}\OperatorTok{;}
\NormalTok{Transfinite Curve }\OperatorTok{\{}\DecValTok{6}\OperatorTok{\}} \OperatorTok{=} \DecValTok{12}\NormalTok{ Using Progression }\DecValTok{1}\OperatorTok{;}
\NormalTok{Transfinite Curve }\OperatorTok{\{}\DecValTok{8}\OperatorTok{\}} \OperatorTok{=} \DecValTok{6}\NormalTok{ Using Progression }\DecValTok{1}\OperatorTok{;}
\NormalTok{Transfinite Curve }\OperatorTok{\{}\DecValTok{9}\OperatorTok{\}} \OperatorTok{=} \DecValTok{6}\NormalTok{ Using Progression }\DecValTok{1}\OperatorTok{;}
\NormalTok{Transfinite Curve }\OperatorTok{\{}\DecValTok{1}\OperatorTok{\}} \OperatorTok{=} \DecValTok{30}\NormalTok{ Using Progression }\DecValTok{1}\OperatorTok{;}
\NormalTok{Transfinite Curve }\OperatorTok{\{}\DecValTok{5}\OperatorTok{\}} \OperatorTok{=} \DecValTok{18}\NormalTok{ Using Progression }\DecValTok{1}\OperatorTok{;}
\NormalTok{Transfinite Curve }\OperatorTok{\{}\DecValTok{2}\OperatorTok{\}} \OperatorTok{=} \DecValTok{4}\NormalTok{ Using Progression }\DecValTok{1}\OperatorTok{;}
\NormalTok{Transfinite Curve }\OperatorTok{\{}\DecValTok{4}\OperatorTok{\}} \OperatorTok{=} \DecValTok{4}\NormalTok{ Using Progression }\DecValTok{1}\OperatorTok{;}
\NormalTok{Transfinite Curve }\OperatorTok{\{}\DecValTok{3}\OperatorTok{\}} \OperatorTok{=} \DecValTok{2}\NormalTok{ Using Progression }\DecValTok{1}\OperatorTok{;}
\end{Highlighting}
\end{Shaded}

\hypertarget{results}{%
\subsection{Results}\label{results}}

\hypertarget{case-1-h-0-m-1}{%
\subsubsection{\texorpdfstring{Case 1:
\(H = 0\ m\)}{Case 1: H = 0\textbackslash{} m}}\label{case-1-h-0-m-1}}

\hypertarget{laminar-flow-v-0.005-ms}{%
\paragraph{\texorpdfstring{Laminar flow
\((v = 0.005\ m/s)\)}{Laminar flow (v = 0.005\textbackslash{} m/s)}}\label{laminar-flow-v-0.005-ms}}

\includegraphics{C:/Users/lolop/Repositories/TeXURL/test/.TeXURL_dump/b2620b16.png} \textbf{Figure 3.1 A:}
Velocity profile diagram under laminar conditions with height of stream
over weir = \(0\ m\)

\includegraphics{C:/Users/lolop/Repositories/TeXURL/test/.TeXURL_dump/b30c0b1a.png} \textbf{Figure 3.1 B:}
Pressure field diagram under laminar conditions with height of stream
over weir = \(0\ m\)

\hypertarget{turbulent-flow-v-5-ms}{%
\paragraph{\texorpdfstring{Turbulent flow
\((v = 5\ m/s)\)}{Turbulent flow (v = 5\textbackslash{} m/s)}}\label{turbulent-flow-v-5-ms}}

\includegraphics{C:/Users/lolop/Repositories/TeXURL/test/.TeXURL_dump/b20d0b09.png} \textbf{Figure 3.2 A:}
Velocity profile diagram under turbulent conditions with height of
stream over weir = \(0\ m\)

\includegraphics{C:/Users/lolop/Repositories/TeXURL/test/.TeXURL_dump/b5fa0b7d.png} \textbf{Figure 3.2 B:}
Pressure field diagram under turbulent conditions with height of stream
over weir = \(0\ m\)

\hypertarget{case-2-h-0.5-m-1}{%
\subsubsection{\texorpdfstring{Case 2:
\(H = 0.5\ m\)}{Case 2: H = 0.5\textbackslash{} m}}\label{case-2-h-0.5-m-1}}

\hypertarget{laminar-flow-v-0.005-ms-1}{%
\paragraph{\texorpdfstring{Laminar flow
\((v = 0.005\ m/s)\)}{Laminar flow (v = 0.005\textbackslash{} m/s)}}\label{laminar-flow-v-0.005-ms-1}}

\includegraphics{C:/Users/lolop/Repositories/TeXURL/test/.TeXURL_dump/b1cf0af8.png} \textbf{Figure 4.1 A:}
Velocity profile diagram under laminar conditions with height of stream
over weir = \(0.5\ m\)

\includegraphics{C:/Users/lolop/Repositories/TeXURL/test/.TeXURL_dump/b05b0aca.png} \textbf{Figure 4.1 B:}
Pressure field diagram under laminar conditions with height of stream
over weir = \(0.5\ m\)

\hypertarget{turbulent-flow-v-5-ms-1}{%
\paragraph{\texorpdfstring{Turbulent flow
\((v = 5\ m/s)\)}{Turbulent flow (v = 5\textbackslash{} m/s)}}\label{turbulent-flow-v-5-ms-1}}

\includegraphics{C:/Users/lolop/Repositories/TeXURL/test/.TeXURL_dump/b2520b25.png} \textbf{Figure 4.2 A:}
Velocity profile diagram under turbulent conditions with height of
stream over weir = \(0.5\ m\)

\includegraphics{C:/Users/lolop/Repositories/TeXURL/test/.TeXURL_dump/b0980ad5.png} \textbf{Figure 4.2 B:}
Pressure field diagram under turbulent conditions with height of stream
over weir = \(0.5\ m\)

\hypertarget{case-3-h-1-m-1}{%
\subsubsection{\texorpdfstring{Case 3:
\(H = 1\ m\)}{Case 3: H = 1\textbackslash{} m}}\label{case-3-h-1-m-1}}

\hypertarget{laminar-flow-v-0.005-ms-2}{%
\paragraph{\texorpdfstring{Laminar flow
\((v = 0.005\ m/s)\)}{Laminar flow (v = 0.005\textbackslash{} m/s)}}\label{laminar-flow-v-0.005-ms-2}}

\includegraphics{C:/Users/lolop/Repositories/TeXURL/test/.TeXURL_dump/b2a70b04.png} \textbf{Figure 5.1 A:}
Velocity profile diagram under laminar conditions with height of stream
over weir = \(1\ m\)

\includegraphics{C:/Users/lolop/Repositories/TeXURL/test/.TeXURL_dump/b3bd0b40.png} \textbf{Figure 5.1 B:}
Pressure field diagram under laminar conditions with height of stream
over weir = \(1\ m\)

\hypertarget{turbulent-flow-v-5-ms-2}{%
\paragraph{\texorpdfstring{Turbulent flow
\((v = 5\ m/s)\)}{Turbulent flow (v = 5\textbackslash{} m/s)}}\label{turbulent-flow-v-5-ms-2}}

\includegraphics{C:/Users/lolop/Repositories/TeXURL/test/.TeXURL_dump/b2300b1c.png} \textbf{Figure 5.2 A:}
Velocity profile diagram under turbulent conditions with height of
stream over weir = \(1\ m\)

\includegraphics{C:/Users/lolop/Repositories/TeXURL/test/.TeXURL_dump/b15b0b01.png} \textbf{Figure 5.2 B:}
Pressure field diagram under turbulent conditions with height of stream
over weir = \(1\ m\)

\hypertarget{conclusion}{%
\subsection{Conclusion}\label{conclusion}}

Weirs are mainly used to control the flow rates of rivers during periods
of high discharge. Sluice gates (or in some cases the height of the weir
crest) can be altered to increase or decrease the volume of water
flowing downstream. As the height of the weir decreases, distance
covered by the fluid over the flat surface also decreases. Thus, for the
optimum outflow conditions the height of weir should be as close to the
height of stream.

    \begin{tcolorbox}[breakable, size=fbox, boxrule=1pt, pad at break*=1mm,colback=cellbackground, colframe=cellborder]
\prompt{In}{incolor}{ }{\boxspacing}
\begin{Verbatim}[commandchars=\\\{\}]

\end{Verbatim}
\end{tcolorbox}


    % Add a bibliography block to the postdoc
    
    
    
\end{document}
